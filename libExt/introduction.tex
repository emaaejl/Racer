Kahoot(it)

should it be context sensitive???


Flow sensitive pointer analysis is required by many program analysis tools. Flow-insensitive analysis is still the dominant pointer analysis that is quite scalable. We are interested to apply flow sensitive analysis across translation units. 


What is points-to graph?


Challenges in cross translation unit:

Flow sensitive pointer analysis requires to have the complete control flow available during the analysis. But, modern program analysis framework like clang frontend works by creating a compiler instance to process each translation unit. Thus, only one translation unit is available during the lifetime of each compiler instance and the current translation unit may have function/method calls from another translation unit. Thus, the pointer analysis need to work with the partial control flow of the programs.

Flow insensitive pointer analysis typically stores points-to graph for each method. On the other hand, flow sensitive pointer analysis requires storing two points-to graph at each node (lots of nodes at each points-to graph), thus store a lots of information, and takes a lot of processing time.  

In order to reduce the flow of information, some techniques like sparse flow of information is used (elaborate it).

  


Approaches to be used by cross translation unit support:


Consider building function summary:


Step 1: Use some flow insensitive points-to analysis to build use-def chain

step 2: use iterative data-flow analysis framework through the use-def chain.
        create halt-point in which function def is not available
        restart analysis from the so called halt-point.
        store points-to graph at some specific program point(L), and between program point compute points-to information relative with the closest Lm point(L), and between program point compute points-to information relative with the closest L.



Implementation in Clang:

           1. steengaard's points-to analysis 
           2. use-def chain (requires implementation)
           3. perform iterative analysis (require analysis) 
